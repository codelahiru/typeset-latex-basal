\documentclass[11pt]{article}
\usepackage{amsfonts, amssymb, amsmath}
\parindent 0pt

\begin{document}

Brackets \\[10pt]
The distributive property states that $a(b+c)=ab+ac$, 
for all $a, b, c \in \mathbb{R}$. \\[6pt]
The equivalence of a is $[a]$.\\[6pt]
The set $A$ is defined to be $\{1,2,3\}$.\\[6pt]
The movie ticket costs $\$10.00$ dollars. \\[6pt]

$$2\left(\frac{1}{x^2+1}\right)$$ \\[2pt]
$$2\left[\frac{1}{x^2+1}\right]$$ \\[2pt]
$$2\left\{\frac{1}{x^2+1}\right\}$$ \\[2pt]
$$2\left\langle  \frac{1}{x^2+1}\right\rangle  $$ \\[2pt]
$$2\left|  \frac{1}{x^2+1}\right|  $$ \\[2pt]
$$\left.\frac{dy}{dx}\right|_{x=1}$$ \\[2pt]
$$1+\left[\frac{1}{1+\left(\frac{1}{1+x}\right)}\right]$$

Tables \\[10pt]
\begin{tabular}{|c|c|c|c|c|c|}
\hline
$x$ &1 &2 &3 &4 &5 \\ \hline
$f(x)$ &10 &11 &12 &13 &14 \\ \hline
\end{tabular}

Arrays
\begin{align}
5x^2-9 = x+3\\
5x^2-x-12 = 0
\end{align}

\begin{align}
5x^2-9 &= x+3\\
5x^2-x-12 &= 0
\end{align}

\begin{align*}
5x^2-9 &= x+3\\
5x^2-x-12 &= 0\\
&= x^2-7x+1
\end{align*}

\begin{align}
5x^2-9 &= x+3\\
5x^2-x-12 &= 0
\end{align}















\end{document}